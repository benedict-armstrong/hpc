\documentclass[unicode,11pt,a4paper,oneside,numbers=endperiod,openany]{scrartcl}

\input{assignment.sty}

\usepackage{minted}
\usepackage{amsmath}
\usepackage{listings}
\usepackage{subcaption}

\begin{document}


\setassignment
\setduedate{Monday 29 April 2024, 23:59 (midnight).}

\serieheader{High-Performance Computing Lab for CSE}{2024}
{Student: Benedict Armstrong}
{Discussed with: Tristan Gabl}{Solution for Project 4}{}
\newline

% \assignmentpolicy

\section{Ring sum using MPI [10 Points]}
For this task we are asked to implement a ring sum using MPI. We calculate the sum of all MPI ranks in a communicator. To do this each process starts by sending its rank to the next process in the ring adds it to an internal sum and then sends the message on to the next process. This is done until each message has made a full loop of the ring. We notice that by the end the sum should be equal to the sum of all ranks in the communicator.

$$
      \sum_{i=0}^{n-1} i = \frac{n(n-1)}{2}
$$


To avoid avoid deadlocks we can make every process send before they receive.


\section{Cartesian domain decomposition and ghost cells exchange [20 Points]}

The implementation of this task was relatively straightforward after doing the MPI tutorial. The main difficulty was not making any errors with the indexing into the data array. The convenient \texttt{MPI\_Cart\_shift} function makes it easy to get the rank of the cartesian up/down/left/right neighbors. To avoid deadlocks I used the asynchronous \texttt{MPI\_Isend} and \texttt{MPI\_Irecv} functions and then waited for all the communication to finish using \texttt{MPI\_Waitall} at the end.
For the diagonal neighbors I used the \texttt{MPI\_Cart\_coords} function to get the coordinates of the current rank and then added/subtracted 1 from each coordinate to get the coordinates of the diagonal neighbors. This was then converted back into a rank using \texttt{MPI\_Cart\_rank}. Running the program using \ref{lst:ghost_run} produces the output in \ref{lst:ghost_output}. For debugging purposes I also printed the neighbors of each rank. Below that we can see that the output data from rank 9 indeed matches the expected result outlined in the assignment.

\begin{listing}[h!t]
      \begin{minted}{bash}
      mpirun -np 16 ./build/ghost
      \end{minted}
      \caption{Running the ghost cell exchange program}
      \label{lst:ghost_run}
\end{listing}

\begin{listing}[h!t]
      \begin{minted}{bash}
      Neighbors of rank 9
      +-----------+
      |  4|  5|  6|
      |-----------|
      |  8|  9| 10|
      |-----------|
      | 12| 13| 14|
      +-----------+
      data of rank 9 after communication
      4  5  5  5  5  5  5  6 
      8  9  9  9  9  9  9 10 
      8  9  9  9  9  9  9 10 
      8  9  9  9  9  9  9 10 
      8  9  9  9  9  9  9 10 
      8  9  9  9  9  9  9 10 
      8  9  9  9  9  9  9 10 
      12 13 13 13 13 13 13 14 
      \end{minted}
      \caption{Output of the ghost cell exchange program}
      \label{lst:ghost_output}
\end{listing}


\section{Parallelizing the Mandelbrot set using MPI [30 Points]}

To start I implemented the partitioning as outlined in the assignment handout. The MPI documentation at \href{https://rookiehpc.org/mpi/docs/} was very helpful for this. After each process has calculated its part of the Mandelbrot set we need to gather the data back to the root process. This is implemented by simply receiving the data from each process in the root process (rank 0) and writing the data to a png file. There is no danger of a deadlock here as there are no dependencies between the processes. Only the root process writes the data to a file. The output image is shown in figure \ref{fig:mandelbrot}.

To evaluate the performance I ran the program with $p = 1, 2, 4, 8, 16$ processes and an image size of $4096 \times 4096$. Figure \ref{fig:mandelbrot_perf} shows the time taken by each processes (rank) for different number of processes.

In Figure \ref{fig:mandelbrot_perf} we can see that for two processes we get a pretty good speedup with both taking around half the time of the serial version. However, as we increase the number of processes the time taken doesn't decrease equally for all processes. This is likely due to the fact that the Mandelbrot set is not evenly distributed across the image. This means that some processes will have more work to do than others. To fix this we would need to implement a better work distribution algorithm. We could start by calculating a low resolution version of the Mandelbrot set and then splitting the chunks based on how many colored pixels we have in each chuck.

\begin{figure}
      \centering
      \includegraphics[width=.5\linewidth]{plots/mandel.png}
      \caption{Mandelbrot set}
      \label{fig:mandelbrot}
\end{figure}

\begin{figure}
      \centering
      \includegraphics[width=.8\linewidth]{plots/mandel_perf.pdf}
      \caption{Performance of the Mandelbrot set program}
      \label{fig:mandelbrot_perf}
\end{figure}


\section{Parallel matrix-vector multiplication and the power method
        [40 Points]}

The last task was to implement the power method for calculating the largest eigenvalue of a matrix. The main changes required from the serial version are the partitioning of the matrix and then the matrix-vector multiplication. The only new concept for this task was using the \texttt{MPI\_Allgatherv} function to gather the data from all processes back to the root process. The code for this be found in \texttt{code/powermethod/powermethod\_rows.c}.

To evaluate the performance I ran the program with $p = 1, 2, 4, 8, 16, 32, 64$ processes and a square matrix with $1e4$ rows for the strong and $3000 * \sqrt{p}$ rows for the weak benchmark. The program was evaluated both on a single node running all processes and multiple nodes with each processes running on a separate node.

TODO: Add discussion of the results


\begin{figure}
      \begin{subfigure}{.5\textwidth}
            \centering
            \includegraphics[width=1\linewidth]{plots/power_strong_scaling.pdf}
            \caption{Strong scaling}
            \label{fig:powermethod_strong}
      \end{subfigure}
      \begin{subfigure}{.5\textwidth}
            \centering
            \includegraphics[width=1\linewidth]{plots/power_strong_speedup.pdf}
            \caption{Strong scaling speedup}
            \label{fig:powermethod_speedup}
      \end{subfigure}
\end{figure}


\begin{figure}
      \begin{subfigure}{.5\textwidth}
            \centering
            \includegraphics[width=1\linewidth]{plots/power_weak_scaling.pdf}
            \caption{Weak scaling}
            \label{fig:powermethod_weak}
      \end{subfigure}
      \begin{subfigure}{.5\textwidth}
            \centering
            \includegraphics[width=1\linewidth]{plots/power_weak_scaling_efficiency.pdf}
            \caption{Weak scaling efficiency}
            \label{fig:powermethod_efficiency}
      \end{subfigure}
\end{figure}

\end{document}
